
% Default to the notebook output style

    


% Inherit from the specified cell style.




    
\documentclass[11pt]{article}

    
    
    \usepackage[T1]{fontenc}
    % Nicer default font (+ math font) than Computer Modern for most use cases
    \usepackage{mathpazo}

    % Basic figure setup, for now with no caption control since it's done
    % automatically by Pandoc (which extracts ![](path) syntax from Markdown).
    \usepackage{graphicx}
    % We will generate all images so they have a width \maxwidth. This means
    % that they will get their normal width if they fit onto the page, but
    % are scaled down if they would overflow the margins.
    \makeatletter
    \def\maxwidth{\ifdim\Gin@nat@width>\linewidth\linewidth
    \else\Gin@nat@width\fi}
    \makeatother
    \let\Oldincludegraphics\includegraphics
    % Set max figure width to be 80% of text width, for now hardcoded.
    \renewcommand{\includegraphics}[1]{\Oldincludegraphics[width=.8\maxwidth]{#1}}
    % Ensure that by default, figures have no caption (until we provide a
    % proper Figure object with a Caption API and a way to capture that
    % in the conversion process - todo).
    \usepackage{caption}
    \DeclareCaptionLabelFormat{nolabel}{}
    \captionsetup{labelformat=nolabel}

    \usepackage{adjustbox} % Used to constrain images to a maximum size 
    \usepackage{xcolor} % Allow colors to be defined
    \usepackage{enumerate} % Needed for markdown enumerations to work
    \usepackage{geometry} % Used to adjust the document margins
    \usepackage{amsmath} % Equations
    \usepackage{amssymb} % Equations
    \usepackage{textcomp} % defines textquotesingle
    % Hack from http://tex.stackexchange.com/a/47451/13684:
    \AtBeginDocument{%
        \def\PYZsq{\textquotesingle}% Upright quotes in Pygmentized code
    }
    \usepackage{upquote} % Upright quotes for verbatim code
    \usepackage{eurosym} % defines \euro
    \usepackage[mathletters]{ucs} % Extended unicode (utf-8) support
    \usepackage[utf8x]{inputenc} % Allow utf-8 characters in the tex document
    \usepackage{fancyvrb} % verbatim replacement that allows latex
    \usepackage{grffile} % extends the file name processing of package graphics 
                         % to support a larger range 
    % The hyperref package gives us a pdf with properly built
    % internal navigation ('pdf bookmarks' for the table of contents,
    % internal cross-reference links, web links for URLs, etc.)
    \usepackage{hyperref}
    \usepackage{longtable} % longtable support required by pandoc >1.10
    \usepackage{booktabs}  % table support for pandoc > 1.12.2
    \usepackage[inline]{enumitem} % IRkernel/repr support (it uses the enumerate* environment)
    \usepackage[normalem]{ulem} % ulem is needed to support strikethroughs (\sout)
                                % normalem makes italics be italics, not underlines
    

    
    
    % Colors for the hyperref package
    \definecolor{urlcolor}{rgb}{0,.145,.698}
    \definecolor{linkcolor}{rgb}{.71,0.21,0.01}
    \definecolor{citecolor}{rgb}{.12,.54,.11}

    % ANSI colors
    \definecolor{ansi-black}{HTML}{3E424D}
    \definecolor{ansi-black-intense}{HTML}{282C36}
    \definecolor{ansi-red}{HTML}{E75C58}
    \definecolor{ansi-red-intense}{HTML}{B22B31}
    \definecolor{ansi-green}{HTML}{00A250}
    \definecolor{ansi-green-intense}{HTML}{007427}
    \definecolor{ansi-yellow}{HTML}{DDB62B}
    \definecolor{ansi-yellow-intense}{HTML}{B27D12}
    \definecolor{ansi-blue}{HTML}{208FFB}
    \definecolor{ansi-blue-intense}{HTML}{0065CA}
    \definecolor{ansi-magenta}{HTML}{D160C4}
    \definecolor{ansi-magenta-intense}{HTML}{A03196}
    \definecolor{ansi-cyan}{HTML}{60C6C8}
    \definecolor{ansi-cyan-intense}{HTML}{258F8F}
    \definecolor{ansi-white}{HTML}{C5C1B4}
    \definecolor{ansi-white-intense}{HTML}{A1A6B2}

    % commands and environments needed by pandoc snippets
    % extracted from the output of `pandoc -s`
    \providecommand{\tightlist}{%
      \setlength{\itemsep}{0pt}\setlength{\parskip}{0pt}}
    \DefineVerbatimEnvironment{Highlighting}{Verbatim}{commandchars=\\\{\}}
    % Add ',fontsize=\small' for more characters per line
    \newenvironment{Shaded}{}{}
    \newcommand{\KeywordTok}[1]{\textcolor[rgb]{0.00,0.44,0.13}{\textbf{{#1}}}}
    \newcommand{\DataTypeTok}[1]{\textcolor[rgb]{0.56,0.13,0.00}{{#1}}}
    \newcommand{\DecValTok}[1]{\textcolor[rgb]{0.25,0.63,0.44}{{#1}}}
    \newcommand{\BaseNTok}[1]{\textcolor[rgb]{0.25,0.63,0.44}{{#1}}}
    \newcommand{\FloatTok}[1]{\textcolor[rgb]{0.25,0.63,0.44}{{#1}}}
    \newcommand{\CharTok}[1]{\textcolor[rgb]{0.25,0.44,0.63}{{#1}}}
    \newcommand{\StringTok}[1]{\textcolor[rgb]{0.25,0.44,0.63}{{#1}}}
    \newcommand{\CommentTok}[1]{\textcolor[rgb]{0.38,0.63,0.69}{\textit{{#1}}}}
    \newcommand{\OtherTok}[1]{\textcolor[rgb]{0.00,0.44,0.13}{{#1}}}
    \newcommand{\AlertTok}[1]{\textcolor[rgb]{1.00,0.00,0.00}{\textbf{{#1}}}}
    \newcommand{\FunctionTok}[1]{\textcolor[rgb]{0.02,0.16,0.49}{{#1}}}
    \newcommand{\RegionMarkerTok}[1]{{#1}}
    \newcommand{\ErrorTok}[1]{\textcolor[rgb]{1.00,0.00,0.00}{\textbf{{#1}}}}
    \newcommand{\NormalTok}[1]{{#1}}
    
    % Additional commands for more recent versions of Pandoc
    \newcommand{\ConstantTok}[1]{\textcolor[rgb]{0.53,0.00,0.00}{{#1}}}
    \newcommand{\SpecialCharTok}[1]{\textcolor[rgb]{0.25,0.44,0.63}{{#1}}}
    \newcommand{\VerbatimStringTok}[1]{\textcolor[rgb]{0.25,0.44,0.63}{{#1}}}
    \newcommand{\SpecialStringTok}[1]{\textcolor[rgb]{0.73,0.40,0.53}{{#1}}}
    \newcommand{\ImportTok}[1]{{#1}}
    \newcommand{\DocumentationTok}[1]{\textcolor[rgb]{0.73,0.13,0.13}{\textit{{#1}}}}
    \newcommand{\AnnotationTok}[1]{\textcolor[rgb]{0.38,0.63,0.69}{\textbf{\textit{{#1}}}}}
    \newcommand{\CommentVarTok}[1]{\textcolor[rgb]{0.38,0.63,0.69}{\textbf{\textit{{#1}}}}}
    \newcommand{\VariableTok}[1]{\textcolor[rgb]{0.10,0.09,0.49}{{#1}}}
    \newcommand{\ControlFlowTok}[1]{\textcolor[rgb]{0.00,0.44,0.13}{\textbf{{#1}}}}
    \newcommand{\OperatorTok}[1]{\textcolor[rgb]{0.40,0.40,0.40}{{#1}}}
    \newcommand{\BuiltInTok}[1]{{#1}}
    \newcommand{\ExtensionTok}[1]{{#1}}
    \newcommand{\PreprocessorTok}[1]{\textcolor[rgb]{0.74,0.48,0.00}{{#1}}}
    \newcommand{\AttributeTok}[1]{\textcolor[rgb]{0.49,0.56,0.16}{{#1}}}
    \newcommand{\InformationTok}[1]{\textcolor[rgb]{0.38,0.63,0.69}{\textbf{\textit{{#1}}}}}
    \newcommand{\WarningTok}[1]{\textcolor[rgb]{0.38,0.63,0.69}{\textbf{\textit{{#1}}}}}
    
    
    % Define a nice break command that doesn't care if a line doesn't already
    % exist.
    \def\br{\hspace*{\fill} \\* }
    % Math Jax compatability definitions
    \def\gt{>}
    \def\lt{<}
    % Document parameters
    \title{hw6}
    
    
    

    % Pygments definitions
    
\makeatletter
\def\PY@reset{\let\PY@it=\relax \let\PY@bf=\relax%
    \let\PY@ul=\relax \let\PY@tc=\relax%
    \let\PY@bc=\relax \let\PY@ff=\relax}
\def\PY@tok#1{\csname PY@tok@#1\endcsname}
\def\PY@toks#1+{\ifx\relax#1\empty\else%
    \PY@tok{#1}\expandafter\PY@toks\fi}
\def\PY@do#1{\PY@bc{\PY@tc{\PY@ul{%
    \PY@it{\PY@bf{\PY@ff{#1}}}}}}}
\def\PY#1#2{\PY@reset\PY@toks#1+\relax+\PY@do{#2}}

\expandafter\def\csname PY@tok@gd\endcsname{\def\PY@tc##1{\textcolor[rgb]{0.63,0.00,0.00}{##1}}}
\expandafter\def\csname PY@tok@gu\endcsname{\let\PY@bf=\textbf\def\PY@tc##1{\textcolor[rgb]{0.50,0.00,0.50}{##1}}}
\expandafter\def\csname PY@tok@gt\endcsname{\def\PY@tc##1{\textcolor[rgb]{0.00,0.27,0.87}{##1}}}
\expandafter\def\csname PY@tok@gs\endcsname{\let\PY@bf=\textbf}
\expandafter\def\csname PY@tok@gr\endcsname{\def\PY@tc##1{\textcolor[rgb]{1.00,0.00,0.00}{##1}}}
\expandafter\def\csname PY@tok@cm\endcsname{\let\PY@it=\textit\def\PY@tc##1{\textcolor[rgb]{0.25,0.50,0.50}{##1}}}
\expandafter\def\csname PY@tok@vg\endcsname{\def\PY@tc##1{\textcolor[rgb]{0.10,0.09,0.49}{##1}}}
\expandafter\def\csname PY@tok@vi\endcsname{\def\PY@tc##1{\textcolor[rgb]{0.10,0.09,0.49}{##1}}}
\expandafter\def\csname PY@tok@vm\endcsname{\def\PY@tc##1{\textcolor[rgb]{0.10,0.09,0.49}{##1}}}
\expandafter\def\csname PY@tok@mh\endcsname{\def\PY@tc##1{\textcolor[rgb]{0.40,0.40,0.40}{##1}}}
\expandafter\def\csname PY@tok@cs\endcsname{\let\PY@it=\textit\def\PY@tc##1{\textcolor[rgb]{0.25,0.50,0.50}{##1}}}
\expandafter\def\csname PY@tok@ge\endcsname{\let\PY@it=\textit}
\expandafter\def\csname PY@tok@vc\endcsname{\def\PY@tc##1{\textcolor[rgb]{0.10,0.09,0.49}{##1}}}
\expandafter\def\csname PY@tok@il\endcsname{\def\PY@tc##1{\textcolor[rgb]{0.40,0.40,0.40}{##1}}}
\expandafter\def\csname PY@tok@go\endcsname{\def\PY@tc##1{\textcolor[rgb]{0.53,0.53,0.53}{##1}}}
\expandafter\def\csname PY@tok@cp\endcsname{\def\PY@tc##1{\textcolor[rgb]{0.74,0.48,0.00}{##1}}}
\expandafter\def\csname PY@tok@gi\endcsname{\def\PY@tc##1{\textcolor[rgb]{0.00,0.63,0.00}{##1}}}
\expandafter\def\csname PY@tok@gh\endcsname{\let\PY@bf=\textbf\def\PY@tc##1{\textcolor[rgb]{0.00,0.00,0.50}{##1}}}
\expandafter\def\csname PY@tok@ni\endcsname{\let\PY@bf=\textbf\def\PY@tc##1{\textcolor[rgb]{0.60,0.60,0.60}{##1}}}
\expandafter\def\csname PY@tok@nl\endcsname{\def\PY@tc##1{\textcolor[rgb]{0.63,0.63,0.00}{##1}}}
\expandafter\def\csname PY@tok@nn\endcsname{\let\PY@bf=\textbf\def\PY@tc##1{\textcolor[rgb]{0.00,0.00,1.00}{##1}}}
\expandafter\def\csname PY@tok@no\endcsname{\def\PY@tc##1{\textcolor[rgb]{0.53,0.00,0.00}{##1}}}
\expandafter\def\csname PY@tok@na\endcsname{\def\PY@tc##1{\textcolor[rgb]{0.49,0.56,0.16}{##1}}}
\expandafter\def\csname PY@tok@nb\endcsname{\def\PY@tc##1{\textcolor[rgb]{0.00,0.50,0.00}{##1}}}
\expandafter\def\csname PY@tok@nc\endcsname{\let\PY@bf=\textbf\def\PY@tc##1{\textcolor[rgb]{0.00,0.00,1.00}{##1}}}
\expandafter\def\csname PY@tok@nd\endcsname{\def\PY@tc##1{\textcolor[rgb]{0.67,0.13,1.00}{##1}}}
\expandafter\def\csname PY@tok@ne\endcsname{\let\PY@bf=\textbf\def\PY@tc##1{\textcolor[rgb]{0.82,0.25,0.23}{##1}}}
\expandafter\def\csname PY@tok@nf\endcsname{\def\PY@tc##1{\textcolor[rgb]{0.00,0.00,1.00}{##1}}}
\expandafter\def\csname PY@tok@si\endcsname{\let\PY@bf=\textbf\def\PY@tc##1{\textcolor[rgb]{0.73,0.40,0.53}{##1}}}
\expandafter\def\csname PY@tok@s2\endcsname{\def\PY@tc##1{\textcolor[rgb]{0.73,0.13,0.13}{##1}}}
\expandafter\def\csname PY@tok@nt\endcsname{\let\PY@bf=\textbf\def\PY@tc##1{\textcolor[rgb]{0.00,0.50,0.00}{##1}}}
\expandafter\def\csname PY@tok@nv\endcsname{\def\PY@tc##1{\textcolor[rgb]{0.10,0.09,0.49}{##1}}}
\expandafter\def\csname PY@tok@s1\endcsname{\def\PY@tc##1{\textcolor[rgb]{0.73,0.13,0.13}{##1}}}
\expandafter\def\csname PY@tok@dl\endcsname{\def\PY@tc##1{\textcolor[rgb]{0.73,0.13,0.13}{##1}}}
\expandafter\def\csname PY@tok@ch\endcsname{\let\PY@it=\textit\def\PY@tc##1{\textcolor[rgb]{0.25,0.50,0.50}{##1}}}
\expandafter\def\csname PY@tok@m\endcsname{\def\PY@tc##1{\textcolor[rgb]{0.40,0.40,0.40}{##1}}}
\expandafter\def\csname PY@tok@gp\endcsname{\let\PY@bf=\textbf\def\PY@tc##1{\textcolor[rgb]{0.00,0.00,0.50}{##1}}}
\expandafter\def\csname PY@tok@sh\endcsname{\def\PY@tc##1{\textcolor[rgb]{0.73,0.13,0.13}{##1}}}
\expandafter\def\csname PY@tok@ow\endcsname{\let\PY@bf=\textbf\def\PY@tc##1{\textcolor[rgb]{0.67,0.13,1.00}{##1}}}
\expandafter\def\csname PY@tok@sx\endcsname{\def\PY@tc##1{\textcolor[rgb]{0.00,0.50,0.00}{##1}}}
\expandafter\def\csname PY@tok@bp\endcsname{\def\PY@tc##1{\textcolor[rgb]{0.00,0.50,0.00}{##1}}}
\expandafter\def\csname PY@tok@c1\endcsname{\let\PY@it=\textit\def\PY@tc##1{\textcolor[rgb]{0.25,0.50,0.50}{##1}}}
\expandafter\def\csname PY@tok@fm\endcsname{\def\PY@tc##1{\textcolor[rgb]{0.00,0.00,1.00}{##1}}}
\expandafter\def\csname PY@tok@o\endcsname{\def\PY@tc##1{\textcolor[rgb]{0.40,0.40,0.40}{##1}}}
\expandafter\def\csname PY@tok@kc\endcsname{\let\PY@bf=\textbf\def\PY@tc##1{\textcolor[rgb]{0.00,0.50,0.00}{##1}}}
\expandafter\def\csname PY@tok@c\endcsname{\let\PY@it=\textit\def\PY@tc##1{\textcolor[rgb]{0.25,0.50,0.50}{##1}}}
\expandafter\def\csname PY@tok@mf\endcsname{\def\PY@tc##1{\textcolor[rgb]{0.40,0.40,0.40}{##1}}}
\expandafter\def\csname PY@tok@err\endcsname{\def\PY@bc##1{\setlength{\fboxsep}{0pt}\fcolorbox[rgb]{1.00,0.00,0.00}{1,1,1}{\strut ##1}}}
\expandafter\def\csname PY@tok@mb\endcsname{\def\PY@tc##1{\textcolor[rgb]{0.40,0.40,0.40}{##1}}}
\expandafter\def\csname PY@tok@ss\endcsname{\def\PY@tc##1{\textcolor[rgb]{0.10,0.09,0.49}{##1}}}
\expandafter\def\csname PY@tok@sr\endcsname{\def\PY@tc##1{\textcolor[rgb]{0.73,0.40,0.53}{##1}}}
\expandafter\def\csname PY@tok@mo\endcsname{\def\PY@tc##1{\textcolor[rgb]{0.40,0.40,0.40}{##1}}}
\expandafter\def\csname PY@tok@kd\endcsname{\let\PY@bf=\textbf\def\PY@tc##1{\textcolor[rgb]{0.00,0.50,0.00}{##1}}}
\expandafter\def\csname PY@tok@mi\endcsname{\def\PY@tc##1{\textcolor[rgb]{0.40,0.40,0.40}{##1}}}
\expandafter\def\csname PY@tok@kn\endcsname{\let\PY@bf=\textbf\def\PY@tc##1{\textcolor[rgb]{0.00,0.50,0.00}{##1}}}
\expandafter\def\csname PY@tok@cpf\endcsname{\let\PY@it=\textit\def\PY@tc##1{\textcolor[rgb]{0.25,0.50,0.50}{##1}}}
\expandafter\def\csname PY@tok@kr\endcsname{\let\PY@bf=\textbf\def\PY@tc##1{\textcolor[rgb]{0.00,0.50,0.00}{##1}}}
\expandafter\def\csname PY@tok@s\endcsname{\def\PY@tc##1{\textcolor[rgb]{0.73,0.13,0.13}{##1}}}
\expandafter\def\csname PY@tok@kp\endcsname{\def\PY@tc##1{\textcolor[rgb]{0.00,0.50,0.00}{##1}}}
\expandafter\def\csname PY@tok@w\endcsname{\def\PY@tc##1{\textcolor[rgb]{0.73,0.73,0.73}{##1}}}
\expandafter\def\csname PY@tok@kt\endcsname{\def\PY@tc##1{\textcolor[rgb]{0.69,0.00,0.25}{##1}}}
\expandafter\def\csname PY@tok@sc\endcsname{\def\PY@tc##1{\textcolor[rgb]{0.73,0.13,0.13}{##1}}}
\expandafter\def\csname PY@tok@sb\endcsname{\def\PY@tc##1{\textcolor[rgb]{0.73,0.13,0.13}{##1}}}
\expandafter\def\csname PY@tok@sa\endcsname{\def\PY@tc##1{\textcolor[rgb]{0.73,0.13,0.13}{##1}}}
\expandafter\def\csname PY@tok@k\endcsname{\let\PY@bf=\textbf\def\PY@tc##1{\textcolor[rgb]{0.00,0.50,0.00}{##1}}}
\expandafter\def\csname PY@tok@se\endcsname{\let\PY@bf=\textbf\def\PY@tc##1{\textcolor[rgb]{0.73,0.40,0.13}{##1}}}
\expandafter\def\csname PY@tok@sd\endcsname{\let\PY@it=\textit\def\PY@tc##1{\textcolor[rgb]{0.73,0.13,0.13}{##1}}}

\def\PYZbs{\char`\\}
\def\PYZus{\char`\_}
\def\PYZob{\char`\{}
\def\PYZcb{\char`\}}
\def\PYZca{\char`\^}
\def\PYZam{\char`\&}
\def\PYZlt{\char`\<}
\def\PYZgt{\char`\>}
\def\PYZsh{\char`\#}
\def\PYZpc{\char`\%}
\def\PYZdl{\char`\$}
\def\PYZhy{\char`\-}
\def\PYZsq{\char`\'}
\def\PYZdq{\char`\"}
\def\PYZti{\char`\~}
% for compatibility with earlier versions
\def\PYZat{@}
\def\PYZlb{[}
\def\PYZrb{]}
\makeatother


    % Exact colors from NB
    \definecolor{incolor}{rgb}{0.0, 0.0, 0.5}
    \definecolor{outcolor}{rgb}{0.545, 0.0, 0.0}



    
    % Prevent overflowing lines due to hard-to-break entities
    \sloppy 
    % Setup hyperref package
    \hypersetup{
      breaklinks=true,  % so long urls are correctly broken across lines
      colorlinks=true,
      urlcolor=urlcolor,
      linkcolor=linkcolor,
      citecolor=citecolor,
      }
    % Slightly bigger margins than the latex defaults
    
    \geometry{verbose,tmargin=1in,bmargin=1in,lmargin=1in,rmargin=1in}
    
    

    \begin{document}
    
    
    \maketitle
    
    

    
    \section{Problem 1}\label{problem-1}

Solve the following problem using KKT conditions
\[Min~~f = 4x_1 - 3x_2 + 2x_1^2 - 3x_1x_2 + 4x_2^2\]
\[g_1(x):~~2x_1 - 1.5x_2 = 5\]

    The KKT conditions can be written as:

\begin{align}
\frac{\partial f}{\partial x_1} - \lambda \frac{\partial g_1}{\partial x_1} = 0\\
\frac{\partial f}{\partial x_2} - \lambda \frac{\partial g_1}{\partial x_2} = 0\\
g_1(x) - b_1 = 0
\end{align}

which evaluates to:

\begin{align}
4x_1 - 3x_2 - 2\lambda = -4\\
-3x_1 + 8x_2 + 1.5\lambda = 3\\
2x_1 - 1.5x_2 - 5 = 0
\end{align}

This can be solved using a system of linear equations

    \begin{Verbatim}[commandchars=\\\{\}]
{\color{incolor}In [{\color{incolor}115}]:} \PY{n}{A} \PY{o}{=} \PY{p}{[}\PY{l+m+mi}{4} \PY{o}{\PYZhy{}}\PY{l+m+mi}{3} \PY{o}{\PYZhy{}}\PY{l+m+mi}{2}\PY{p}{;} \PY{o}{\PYZhy{}}\PY{l+m+mi}{3} \PY{l+m+mi}{8} \PY{l+m+mf}{1.5}\PY{p}{;} \PY{l+m+mi}{2} \PY{o}{\PYZhy{}}\PY{l+m+mf}{1.5} \PY{l+m+mi}{0}\PY{p}{]}\PY{p}{;}
          \PY{n}{b} \PY{o}{=} \PY{p}{[}\PY{o}{\PYZhy{}}\PY{l+m+mi}{4}\PY{p}{;} \PY{l+m+mi}{3}\PY{p}{;} \PY{l+m+mi}{5}\PY{p}{]}\PY{p}{;}
          \PY{n}{c} \PY{o}{=} \PY{n}{A}\PY{o}{\PYZbs{}}\PY{n}{b}
          \PY{n}{println}\PY{p}{(}\PY{n}{c}\PY{p}{)}
          
          \PY{n}{fun}\PY{p}{(}\PY{n}{x1}\PY{p}{,} \PY{n}{x2}\PY{p}{)} \PY{o}{=} \PY{l+m+mi}{4}\PY{n}{x1} \PY{o}{\PYZhy{}}\PY{l+m+mi}{3}\PY{n}{x2} \PY{o}{+} \PY{l+m+mi}{2}\PY{o}{*}\PY{n}{x1}\PY{o}{\PYZca{}}\PY{l+m+mi}{2} \PY{o}{\PYZhy{}} \PY{l+m+mi}{3}\PY{n}{x1}\PY{o}{*}\PY{n}{x2} \PY{o}{+} \PY{l+m+mi}{4} \PY{o}{*} \PY{n}{x2}\PY{o}{\PYZca{}}\PY{l+m+mi}{2}
          \PY{n}{opt} \PY{o}{=} \PY{n}{fun}\PY{p}{(}\PY{n}{c}\PY{p}{[}\PY{l+m+mi}{1}\PY{p}{]}\PY{p}{,} \PY{n}{c}\PY{p}{[}\PY{l+m+mi}{2}\PY{p}{]}\PY{p}{)}
          \PY{n}{println}\PY{p}{(}\PY{l+s}{\PYZdq{}}\PY{l+s}{O}\PY{l+s}{p}\PY{l+s}{t}\PY{l+s}{i}\PY{l+s}{m}\PY{l+s}{u}\PY{l+s}{m}\PY{l+s}{:}\PY{l+s}{ }\PY{l+s}{\PYZdq{}}\PY{p}{,} \PY{n}{opt}\PY{p}{)}
\end{Verbatim}


    \begin{Verbatim}[commandchars=\\\{\}]
[2.5, 0.0, 7.0]
Optimum: 22.5

    \end{Verbatim}

    The Values then are:

\begin{align}
x_1 = 2.5\\
x_2 = 0\\
\lambda = 7
\end{align}

which gives the objective \[ f = 22.5\]

    \subsubsection{(b) Change the constraint to
be:}\label{b-change-the-constraint-to-be}

\[g_1(x) = 2x_1 - 1.5x_2 = 5.1\] because the change in the constraint is
0.1 we expect that the change in the objective will be
\(0.1 * \lambda = 0.7\)

    \begin{Verbatim}[commandchars=\\\{\}]
{\color{incolor}In [{\color{incolor}116}]:} \PY{n}{A} \PY{o}{=} \PY{p}{[}\PY{l+m+mi}{4} \PY{o}{\PYZhy{}}\PY{l+m+mi}{3} \PY{o}{\PYZhy{}}\PY{l+m+mi}{2}\PY{p}{;} \PY{o}{\PYZhy{}}\PY{l+m+mi}{3} \PY{l+m+mi}{8} \PY{l+m+mf}{1.5}\PY{p}{;} \PY{l+m+mi}{2} \PY{o}{\PYZhy{}}\PY{l+m+mf}{1.5} \PY{l+m+mi}{0}\PY{p}{]}\PY{p}{;}
          \PY{n}{b2} \PY{o}{=} \PY{p}{[}\PY{o}{\PYZhy{}}\PY{l+m+mi}{4}\PY{p}{;} \PY{l+m+mi}{3}\PY{p}{;} \PY{l+m+mf}{5.1}\PY{p}{]}\PY{p}{;}
          \PY{n}{c2} \PY{o}{=} \PY{n}{A}\PY{o}{\PYZbs{}}\PY{n}{b2}
          \PY{n}{print}\PY{p}{(}\PY{n}{c2}\PY{p}{)}
          
          \PY{n}{fun}\PY{p}{(}\PY{n}{x1}\PY{p}{,} \PY{n}{x2}\PY{p}{)} \PY{o}{=} \PY{l+m+mi}{4}\PY{n}{x1} \PY{o}{\PYZhy{}}\PY{l+m+mi}{3}\PY{n}{x2} \PY{o}{+} \PY{l+m+mi}{2}\PY{o}{*}\PY{n}{x1}\PY{o}{\PYZca{}}\PY{l+m+mi}{2} \PY{o}{\PYZhy{}} \PY{l+m+mi}{3}\PY{n}{x1}\PY{o}{*}\PY{n}{x2} \PY{o}{+} \PY{l+m+mi}{4} \PY{o}{*} \PY{n}{x2}\PY{o}{\PYZca{}}\PY{l+m+mi}{2}
          \PY{n}{opt2} \PY{o}{=} \PY{n}{fun}\PY{p}{(}\PY{n}{c2}\PY{p}{[}\PY{l+m+mi}{1}\PY{p}{]}\PY{p}{,} \PY{n}{c2}\PY{p}{[}\PY{l+m+mi}{2}\PY{p}{]}\PY{p}{)}
          \PY{n}{println}\PY{p}{(}\PY{l+s}{\PYZdq{}}\PY{l+s+se}{\PYZbs{}n}\PY{l+s}{N}\PY{l+s}{e}\PY{l+s}{w}\PY{l+s}{ }\PY{l+s}{o}\PY{l+s}{p}\PY{l+s}{t}\PY{l+s}{i}\PY{l+s}{m}\PY{l+s}{u}\PY{l+s}{m}\PY{l+s}{:}\PY{l+s}{ }\PY{l+s}{\PYZdq{}}\PY{p}{,} \PY{n}{opt}\PY{p}{)}
          \PY{n}{delta\PYZus{}opt} \PY{o}{=} \PY{n}{opt2} \PY{o}{\PYZhy{}} \PY{n}{opt}
          \PY{n}{println}\PY{p}{(}\PY{l+s}{\PYZdq{}}\PY{l+s}{C}\PY{l+s}{h}\PY{l+s}{a}\PY{l+s}{n}\PY{l+s}{g}\PY{l+s}{e}\PY{l+s}{ }\PY{l+s}{i}\PY{l+s}{n}\PY{l+s}{ }\PY{l+s}{o}\PY{l+s}{p}\PY{l+s}{t}\PY{l+s}{i}\PY{l+s}{m}\PY{l+s}{u}\PY{l+s}{m}\PY{l+s}{:}\PY{l+s}{ }\PY{l+s}{\PYZdq{}}\PY{p}{,} \PY{n}{delta\PYZus{}opt}\PY{p}{)}
\end{Verbatim}


    \begin{Verbatim}[commandchars=\\\{\}]
[2.55, 0.0, 7.1]
New optimum: 22.5
Change in optimum: 0.7049999999999983

    \end{Verbatim}

    As we can see the change in the optimum value was 0.705 which is very
close to the predicted value of 0.7. This shows that the \(\lambda\)
value accurately predicts the change in the optimum.

    \subsubsection{(c) Are the KKT equations for a problem with quadratic
objective and a linear equality constraint always linear? Is this true
for a problem with a quadratic objective and a linear inequality
constraint?}\label{c-are-the-kkt-equations-for-a-problem-with-quadratic-objective-and-a-linear-equality-constraint-always-linear-is-this-true-for-a-problem-with-a-quadratic-objective-and-a-linear-inequality-constraint}

    If the problem has a quadratic objective and a linear equality constrant
then the KKT equations will be linear, if a linear inequality constraint
is present than the problem will also be linear.

    \subsection{Problem 3: Solve the following problem using KKT
conditions}\label{problem-3-solve-the-following-problem-using-kkt-conditions}

\[Min~f(x) = x_1^2 + 2x_2^2 + 3x_3^2\] \[g_1(x): x_1 + 5x_2 = 12\]
\[g_2(x): -2x_1-x_2 - 4x_3 \leq -18\]

    First we must formulate the problem to match the required format. We
change \(g_2\) to be: \[g_2(x): 2x_1 - x_2 + 4x_3 - 18 \leq 0\] which in
turn formulates the system of linear equations representing the KKT
Conditions:

\begin{align}
\begin{bmatrix}
 2&  0&  0&  -1& 2\\ 
 0&  4&  0&  -5& -1\\ 
 0&  0&  6&  0& 4\\ 
 2&  -1&  4&  0& 0\\ 
 1&  5&  0&  0& 0
\end{bmatrix} * 
\begin{bmatrix}
x_1 \\
x_2 \\
x_3 \\
\lambda_2 \\
\lambda_1 
\end{bmatrix} = 
\begin{bmatrix}
0\\0\\0\\18\\12
\end{bmatrix}
\end{align}

    \begin{Verbatim}[commandchars=\\\{\}]
{\color{incolor}In [{\color{incolor}124}]:} \PY{n}{A} \PY{o}{=} \PY{p}{[}\PY{l+m+mi}{2} \PY{l+m+mi}{0} \PY{l+m+mi}{0} \PY{o}{\PYZhy{}}\PY{l+m+mi}{1} \PY{l+m+mi}{2}
               \PY{l+m+mi}{0} \PY{l+m+mi}{4} \PY{l+m+mi}{0} \PY{o}{\PYZhy{}}\PY{l+m+mi}{5} \PY{o}{\PYZhy{}}\PY{l+m+mi}{1}
               \PY{l+m+mi}{0} \PY{l+m+mi}{0} \PY{l+m+mi}{6} \PY{l+m+mi}{0} \PY{l+m+mi}{4}
               \PY{l+m+mi}{2} \PY{o}{\PYZhy{}}\PY{l+m+mi}{1} \PY{l+m+mi}{4} \PY{l+m+mi}{0} \PY{l+m+mi}{0}
               \PY{l+m+mi}{1} \PY{l+m+mi}{5} \PY{l+m+mi}{0} \PY{l+m+mi}{0} \PY{l+m+mi}{0}\PY{p}{]}
          \PY{n}{b} \PY{o}{=} \PY{p}{[}\PY{l+m+mi}{0}
               \PY{l+m+mi}{0}
               \PY{l+m+mi}{0}
               \PY{l+m+mi}{18}
               \PY{l+m+mi}{12}\PY{p}{]}
          \PY{n}{x} \PY{o}{=} \PY{n}{A}\PY{o}{\PYZbs{}}\PY{n}{b}
          \PY{n}{println}\PY{p}{(}\PY{l+s}{\PYZdq{}}\PY{l+s}{T}\PY{l+s}{h}\PY{l+s}{e}\PY{l+s}{ }\PY{l+s}{v}\PY{l+s}{a}\PY{l+s}{l}\PY{l+s}{u}\PY{l+s}{e}\PY{l+s}{s}\PY{l+s}{ }\PY{l+s}{o}\PY{l+s}{f}\PY{l+s}{ }\PY{l+s}{t}\PY{l+s}{h}\PY{l+s}{e}\PY{l+s}{ }\PY{l+s}{x}\PY{l+s}{ }\PY{l+s}{v}\PY{l+s}{e}\PY{l+s}{c}\PY{l+s}{t}\PY{l+s}{o}\PY{l+s}{r}\PY{l+s}{ }\PY{l+s}{a}\PY{l+s}{r}\PY{l+s}{e}\PY{l+s}{:}\PY{l+s}{ }\PY{l+s+si}{\PYZdl{}x}\PY{l+s}{\PYZdq{}}\PY{p}{)}
          \PY{n}{g2}\PY{p}{(}\PY{n}{x1}\PY{p}{,} \PY{n}{x2}\PY{p}{,} \PY{n}{x3}\PY{p}{)} \PY{o}{=} \PY{l+m+mi}{2}\PY{o}{*}\PY{n}{x1} \PY{o}{\PYZhy{}} \PY{n}{x2} \PY{o}{+} \PY{l+m+mi}{4}\PY{n}{x3} \PY{o}{\PYZhy{}} \PY{l+m+mi}{18}
          \PY{n}{g1}\PY{p}{(}\PY{n}{x1}\PY{p}{,} \PY{n}{x2}\PY{p}{)} \PY{o}{=} \PY{n}{x1} \PY{o}{+} \PY{l+m+mi}{5}\PY{n}{x2} \PY{o}{\PYZhy{}} \PY{l+m+mi}{12}
          \PY{n}{g1\PYZus{}x} \PY{o}{=} \PY{n}{g1}\PY{p}{(}\PY{n}{x}\PY{p}{[}\PY{l+m+mi}{1}\PY{p}{]}\PY{p}{,} \PY{n}{x}\PY{p}{[}\PY{l+m+mi}{2}\PY{p}{]}\PY{p}{)}
          \PY{n}{g2\PYZus{}x} \PY{o}{=} \PY{n}{g2}\PY{p}{(}\PY{n}{x}\PY{p}{[}\PY{l+m+mi}{1}\PY{p}{]}\PY{p}{,} \PY{n}{x}\PY{p}{[}\PY{l+m+mi}{2}\PY{p}{]}\PY{p}{,} \PY{n}{x}\PY{p}{[}\PY{l+m+mi}{3}\PY{p}{]}\PY{p}{)}
          \PY{n}{println}\PY{p}{(}\PY{l+s}{\PYZdq{}}\PY{l+s}{g}\PY{l+s}{1}\PY{l+s}{ }\PY{l+s}{e}\PY{l+s}{v}\PY{l+s}{a}\PY{l+s}{l}\PY{l+s}{u}\PY{l+s}{a}\PY{l+s}{t}\PY{l+s}{e}\PY{l+s}{s}\PY{l+s}{ }\PY{l+s}{t}\PY{l+s}{o}\PY{l+s}{:}\PY{l+s}{ }\PY{l+s+si}{\PYZdl{}g1\PYZus{}x}\PY{l+s}{\PYZdq{}}\PY{p}{)}
          \PY{n}{println}\PY{p}{(}\PY{l+s}{\PYZdq{}}\PY{l+s}{g}\PY{l+s}{2}\PY{l+s}{ }\PY{l+s}{e}\PY{l+s}{v}\PY{l+s}{a}\PY{l+s}{l}\PY{l+s}{u}\PY{l+s}{a}\PY{l+s}{t}\PY{l+s}{e}\PY{l+s}{s}\PY{l+s}{ }\PY{l+s}{t}\PY{l+s}{o}\PY{l+s}{:}\PY{l+s}{ }\PY{l+s+si}{\PYZdl{}g2\PYZus{}x}\PY{l+s}{\PYZdq{}}\PY{p}{)}
\end{Verbatim}


    \begin{Verbatim}[commandchars=\\\{\}]
The values of the x vector are: [4.71698, 1.4566, 2.50566, 1.91698, -3.75849]
g1 evaluates to: 3.907985046680551e-14
g2 evaluates to: 3.552713678800501e-15

    \end{Verbatim}

    Here we can see that: \[x_1 = 4.717\\
x_2 = 1.457\\
x_3 = 2.506\\
\lambda_1 = -3.7585\\
\lambda_2 = 1.9170\] We also note from the code that both of the
constraints are binding. This satisfies the neccessary conditions for an
optimum

To check the sufficient conditions we must check that
\(\nabla_x^2L(\textbf{x}^*, \lambda^*)\) is positive definite

\[\nabla_x^2L(\textbf{x}^*, \lambda^*) = \begin{bmatrix} 1&0&0\\ 0&4&0\\ 0&0&6 \end{bmatrix}
 - \lambda_1 \begin{bmatrix} 0&0&0\\0&0&0\\0&0&0\end{bmatrix} 
 - \lambda_2 \begin{bmatrix} 0&0&0\\0&0&0\\0&0&0\end{bmatrix} = \begin{bmatrix} 1&0&0\\ 0&4&0\\ 0&0&6 \end{bmatrix}\]

    \begin{Verbatim}[commandchars=\\\{\}]
{\color{incolor}In [{\color{incolor}126}]:} \PY{n}{L} \PY{o}{=} \PY{p}{[}\PY{l+m+mi}{1} \PY{l+m+mi}{0} \PY{l+m+mi}{0}
              \PY{l+m+mi}{0} \PY{l+m+mi}{4} \PY{l+m+mi}{0}
              \PY{l+m+mi}{0} \PY{l+m+mi}{0} \PY{l+m+mi}{6}\PY{p}{]}
          \PY{n}{isOpt} \PY{o}{=} \PY{n}{isposdef}\PY{p}{(}\PY{n}{L}\PY{p}{)}
          \PY{k}{if} \PY{n}{isOpt}
              \PY{n}{println}\PY{p}{(}\PY{l+s}{\PYZdq{}}\PY{l+s}{T}\PY{l+s}{h}\PY{l+s}{e}\PY{l+s}{ }\PY{l+s}{v}\PY{l+s}{a}\PY{l+s}{l}\PY{l+s}{u}\PY{l+s}{e}\PY{l+s}{s}\PY{l+s}{ }\PY{l+s}{c}\PY{l+s}{o}\PY{l+s}{r}\PY{l+s}{r}\PY{l+s}{e}\PY{l+s}{s}\PY{l+s}{p}\PY{l+s}{o}\PY{l+s}{n}\PY{l+s}{d}\PY{l+s}{ }\PY{l+s}{t}\PY{l+s}{o}\PY{l+s}{ }\PY{l+s}{a}\PY{l+s}{ }\PY{l+s}{c}\PY{l+s}{o}\PY{l+s}{n}\PY{l+s}{s}\PY{l+s}{t}\PY{l+s}{r}\PY{l+s}{a}\PY{l+s}{i}\PY{l+s}{n}\PY{l+s}{e}\PY{l+s}{d}\PY{l+s}{ }\PY{l+s}{o}\PY{l+s}{p}\PY{l+s}{t}\PY{l+s}{i}\PY{l+s}{m}\PY{l+s}{u}\PY{l+s}{m}\PY{l+s}{\PYZdq{}}\PY{p}{)}
          \PY{k}{end}
\end{Verbatim}


    \begin{Verbatim}[commandchars=\\\{\}]
The values correspond to a constrained optimum

    \end{Verbatim}

    because the hessian of the lagrangian function is postive definite then
we can conclude that the point is a constrained minimum.

    \subsection{Problem 4}\label{problem-4}

\[Min~f(x) = x_1^2 + x_2^2\\
g_1(x)=x_1^2+x_2^2 - 9 = 0\\
g_2(x)=x_1+x_2^2-1\leq 0\\
g_3(x)=x_1+x_2-1\leq 0\] Once again the inequality constraints must be
reformatted \[g_2(x)=-x_1-x_2^2+1\geq 0\\
g_3(x)=-x_1-x_2+1\geq 0\]

    \subsubsection{(a) Verify that {[}-2.3723, -1.8364{]} is a local
optimum}\label{a-verify-that--2.3723--1.8364-is-a-local-optimum}

Check which constraints are binding

    \begin{Verbatim}[commandchars=\\\{\}]
{\color{incolor}In [{\color{incolor}119}]:} \PY{n}{x} \PY{o}{=} \PY{p}{[}\PY{o}{\PYZhy{}}\PY{l+m+mf}{2.3723} \PY{o}{\PYZhy{}}\PY{l+m+mf}{1.8364}\PY{p}{]}
          \PY{n}{g1}\PY{p}{(}\PY{n}{x}\PY{p}{)} \PY{o}{=} \PY{n}{x}\PY{p}{[}\PY{l+m+mi}{1}\PY{p}{]}\PY{o}{\PYZca{}}\PY{l+m+mi}{2} \PY{o}{+} \PY{n}{x}\PY{p}{[}\PY{l+m+mi}{2}\PY{p}{]}\PY{o}{\PYZca{}}\PY{l+m+mi}{2} \PY{o}{\PYZhy{}} \PY{l+m+mi}{9}
          \PY{n}{g2}\PY{p}{(}\PY{n}{x}\PY{p}{)} \PY{o}{=} \PY{n}{x}\PY{p}{[}\PY{l+m+mi}{1}\PY{p}{]} \PY{o}{+} \PY{n}{x}\PY{p}{[}\PY{l+m+mi}{2}\PY{p}{]}\PY{o}{\PYZca{}}\PY{l+m+mi}{2} \PY{o}{\PYZhy{}} \PY{l+m+mi}{1}
          \PY{n}{g3}\PY{p}{(}\PY{n}{x}\PY{p}{)} \PY{o}{=} \PY{n}{x}\PY{p}{[}\PY{l+m+mi}{1}\PY{p}{]} \PY{o}{+} \PY{n}{x}\PY{p}{[}\PY{l+m+mi}{2}\PY{p}{]} \PY{o}{\PYZhy{}} \PY{l+m+mi}{1}
          \PY{n}{g1x} \PY{o}{=} \PY{n}{g1}\PY{p}{(}\PY{n}{x}\PY{p}{)}
          \PY{n}{g2x} \PY{o}{=} \PY{n}{g2}\PY{p}{(}\PY{n}{x}\PY{p}{)}
          \PY{n}{g3x} \PY{o}{=} \PY{n}{g3}\PY{p}{(}\PY{n}{x}\PY{p}{)}
          \PY{n}{println}\PY{p}{(}\PY{l+s}{\PYZdq{}}\PY{l+s}{g}\PY{l+s}{1}\PY{l+s}{ }\PY{l+s}{=}\PY{l+s}{ }\PY{l+s+si}{\PYZdl{}g1x}\PY{l+s}{\PYZdq{}}\PY{p}{)}
          \PY{n}{println}\PY{p}{(}\PY{l+s}{\PYZdq{}}\PY{l+s}{g}\PY{l+s}{2}\PY{l+s}{ }\PY{l+s}{=}\PY{l+s}{ }\PY{l+s+si}{\PYZdl{}g2x}\PY{l+s}{\PYZdq{}}\PY{p}{)}
          \PY{n}{println}\PY{p}{(}\PY{l+s}{\PYZdq{}}\PY{l+s}{g}\PY{l+s}{3}\PY{l+s}{ }\PY{l+s}{=}\PY{l+s}{ }\PY{l+s+si}{\PYZdl{}g3x}\PY{l+s}{\PYZdq{}}\PY{p}{)}
\end{Verbatim}


    \begin{Verbatim}[commandchars=\\\{\}]
g1 = 0.00017225000000031798
g2 = 6.49600000000028e-5
g3 = -5.2087

    \end{Verbatim}

    within acceptable roundoff g1 and g2 are binding constraints. We must
now check if the KKT conditions are satisfied.

    \begin{Verbatim}[commandchars=\\\{\}]
{\color{incolor}In [{\color{incolor}128}]:} \PY{k}{using} \PY{n}{NLsolve}
          \PY{k}{function} \PY{n}{f!}\PY{p}{(}\PY{n}{F}\PY{p}{,} \PY{n}{x}\PY{p}{)}
              \PY{n}{F}\PY{p}{[}\PY{l+m+mi}{1}\PY{p}{]} \PY{o}{=} \PY{l+m+mi}{2}\PY{o}{*}\PY{n}{x}\PY{p}{[}\PY{l+m+mi}{1}\PY{p}{]} \PY{o}{\PYZhy{}} \PY{n}{x}\PY{p}{[}\PY{l+m+mi}{3}\PY{p}{]} \PY{o}{*} \PY{l+m+mf}{2.0} \PY{o}{*} \PY{n}{x}\PY{p}{[}\PY{l+m+mi}{1}\PY{p}{]} \PY{o}{+} \PY{n}{x}\PY{p}{[}\PY{l+m+mi}{4}\PY{p}{]}
              \PY{n}{F}\PY{p}{[}\PY{l+m+mi}{2}\PY{p}{]} \PY{o}{=} \PY{l+m+mf}{1.0} \PY{o}{\PYZhy{}} \PY{n}{x}\PY{p}{[}\PY{l+m+mi}{3}\PY{p}{]} \PY{o}{*} \PY{l+m+mf}{2.0} \PY{o}{*} \PY{n}{x}\PY{p}{[}\PY{l+m+mi}{2}\PY{p}{]} \PY{o}{\PYZhy{}} \PY{n}{x}\PY{p}{[}\PY{l+m+mi}{4}\PY{p}{]} \PY{o}{*} \PY{o}{\PYZhy{}}\PY{l+m+mi}{2} \PY{o}{*} \PY{n}{x}\PY{p}{[}\PY{l+m+mi}{2}\PY{p}{]}
              \PY{n}{F}\PY{p}{[}\PY{l+m+mi}{3}\PY{p}{]} \PY{o}{=} \PY{n}{x}\PY{p}{[}\PY{l+m+mi}{1}\PY{p}{]}\PY{o}{\PYZca{}}\PY{l+m+mf}{2.0} \PY{o}{+} \PY{n}{x}\PY{p}{[}\PY{l+m+mi}{2}\PY{p}{]}\PY{o}{\PYZca{}}\PY{l+m+mf}{2.0} \PY{o}{\PYZhy{}} \PY{l+m+mf}{9.0}
              \PY{n}{F}\PY{p}{[}\PY{l+m+mi}{4}\PY{p}{]} \PY{o}{=} \PY{o}{\PYZhy{}}\PY{n}{x}\PY{p}{[}\PY{l+m+mi}{1}\PY{p}{]} \PY{o}{\PYZhy{}} \PY{n}{x}\PY{p}{[}\PY{l+m+mi}{2}\PY{p}{]}\PY{o}{\PYZca{}}\PY{l+m+mf}{2.0} \PY{o}{+} \PY{l+m+mf}{1.0}
          \PY{k}{end}
          
          \PY{n}{x0} \PY{o}{=} \PY{p}{[}\PY{o}{\PYZhy{}}\PY{l+m+mf}{2.37}\PY{p}{;} \PY{o}{\PYZhy{}}\PY{l+m+mf}{1.836}\PY{p}{;} \PY{o}{\PYZhy{}}\PY{l+m+mf}{1.0}\PY{p}{;} \PY{o}{\PYZhy{}}\PY{l+m+mf}{1.0}\PY{p}{]}
          \PY{n}{sol} \PY{o}{=} \PY{n}{nlsolve}\PY{p}{(}\PY{n}{f!}\PY{p}{,} \PY{n}{x0}\PY{p}{)}
          \PY{n}{λ₁} \PY{o}{=} \PY{n}{sol}\PY{o}{.}\PY{n}{zero}\PY{p}{[}\PY{l+m+mi}{3}\PY{p}{]}
          \PY{n}{λ₂} \PY{o}{=} \PY{n}{sol}\PY{o}{.}\PY{n}{zero}\PY{p}{[}\PY{l+m+mi}{4}\PY{p}{]}
          \PY{n}{println}\PY{p}{(}\PY{l+s}{\PYZdq{}}\PY{l+s}{λ}\PY{l+s}{₁}\PY{l+s}{ }\PY{l+s}{=}\PY{l+s}{ }\PY{l+s+si}{\PYZdl{}λ₁}\PY{l+s}{\PYZdq{}}\PY{p}{)}
          \PY{n}{println}\PY{p}{(}\PY{l+s}{\PYZdq{}}\PY{l+s}{λ}\PY{l+s}{₂}\PY{l+s}{ }\PY{l+s}{=}\PY{l+s}{ }\PY{l+s+si}{\PYZdl{}λ₂}\PY{l+s}{\PYZdq{}}\PY{p}{)}
\end{Verbatim}


    \begin{Verbatim}[commandchars=\\\{\}]
λ₁ = 0.7785253137160885
λ₂ = 1.0508005262435787

    \end{Verbatim}

    Because both \(\lambda\) values are positive the KKT conditions are
satisfied. We now check the sufficient conditions.

    \begin{Verbatim}[commandchars=\\\{\}]
{\color{incolor}In [{\color{incolor}121}]:} \PY{n}{hessf} \PY{o}{=} \PY{p}{[}\PY{l+m+mi}{2} \PY{l+m+mi}{0}\PY{p}{;} \PY{l+m+mi}{0} \PY{l+m+mi}{0}\PY{p}{]}
          \PY{n}{hessg1} \PY{o}{=} \PY{p}{[}\PY{l+m+mi}{2} \PY{l+m+mi}{0}\PY{p}{;} \PY{l+m+mi}{0} \PY{l+m+mi}{2}\PY{p}{]}
          \PY{n}{hessg2} \PY{o}{=} \PY{p}{[}\PY{l+m+mi}{0} \PY{l+m+mi}{0}\PY{p}{;} \PY{l+m+mi}{0} \PY{o}{\PYZhy{}}\PY{l+m+mi}{2}\PY{p}{]}
          \PY{n}{lagrangian} \PY{o}{=} \PY{n}{hessf} \PY{o}{\PYZhy{}} \PY{n}{λ₁} \PY{o}{*} \PY{n}{hessg1} \PY{o}{\PYZhy{}} \PY{n}{λ₂} \PY{o}{*} \PY{n}{hessg2}
          \PY{n}{println}\PY{p}{(}\PY{n}{lagrangian}\PY{p}{)}
          
          \PY{n}{isposdef}\PY{p}{(}\PY{n}{lagrangian}\PY{p}{)}
\end{Verbatim}


    \begin{Verbatim}[commandchars=\\\{\}]
[0.442949 0.0; 0.0 0.54455]

    \end{Verbatim}

\begin{Verbatim}[commandchars=\\\{\}]
{\color{outcolor}Out[{\color{outcolor}121}]:} true
\end{Verbatim}
            
    Because the lagrangain function is positive definite the point is a
constrained optimum

\subsubsection{Verify that {[}-2.5, -1.6583{]} is not a local
optimum}\label{verify-that--2.5--1.6583-is-not-a-local-optimum}

    \begin{Verbatim}[commandchars=\\\{\}]
{\color{incolor}In [{\color{incolor}122}]:} \PY{n}{x} \PY{o}{=} \PY{p}{[}\PY{o}{\PYZhy{}}\PY{l+m+mf}{2.5}\PY{p}{,} \PY{o}{\PYZhy{}}\PY{l+m+mf}{1.6583}\PY{p}{]}
          \PY{n}{g1x} \PY{o}{=} \PY{n}{g1}\PY{p}{(}\PY{n}{x}\PY{p}{)}
          \PY{n}{g2x} \PY{o}{=} \PY{n}{g2}\PY{p}{(}\PY{n}{x}\PY{p}{)}
          \PY{n}{g3x} \PY{o}{=} \PY{n}{g3}\PY{p}{(}\PY{n}{x}\PY{p}{)}
          \PY{n}{println}\PY{p}{(}\PY{l+s}{\PYZdq{}}\PY{l+s}{g}\PY{l+s}{1}\PY{l+s}{ }\PY{l+s}{=}\PY{l+s}{ }\PY{l+s+si}{\PYZdl{}g1x}\PY{l+s}{\PYZdq{}}\PY{p}{)}
          \PY{n}{println}\PY{p}{(}\PY{l+s}{\PYZdq{}}\PY{l+s}{g}\PY{l+s}{2}\PY{l+s}{ }\PY{l+s}{=}\PY{l+s}{ }\PY{l+s+si}{\PYZdl{}g2x}\PY{l+s}{\PYZdq{}}\PY{p}{)}
          \PY{n}{println}\PY{p}{(}\PY{l+s}{\PYZdq{}}\PY{l+s}{g}\PY{l+s}{3}\PY{l+s}{ }\PY{l+s}{=}\PY{l+s}{ }\PY{l+s+si}{\PYZdl{}g3x}\PY{l+s}{\PYZdq{}}\PY{p}{)}
\end{Verbatim}


    \begin{Verbatim}[commandchars=\\\{\}]
g1 = -4.1109999999733304e-5
g2 = -0.7500411099999997
g3 = -5.1583000000000006

    \end{Verbatim}

    The only binding constraint is g1, but all of the constraints are
feasible. we now solve for \(\lambda_1\) \[2x_1 - \lambda_1(2x_1) = 0\\
\lambda_1 = -2\\
and\\
1 - \lambda_1(2x_2) = 0\\
\lambda_1 = 3.3\] because we cannot solve for a value of \(\lambda_1\)
the point is not a local optimum

    \subsubsection{Drop the equality constraint from the problem. Using the
countour plot above to see where the optimum lies, solve for the optimum
using the KKT
conditions.}\label{drop-the-equality-constraint-from-the-problem.-using-the-countour-plot-above-to-see-where-the-optimum-lies-solve-for-the-optimum-using-the-kkt-conditions.}

We see that only \(g_2\) is binding the system of equations that we need
to solve is then \[2x_1 + \lambda_2(1) = 0\\
1 + \lambda_2(2x_2) = 0\\
-x_1 - x_2^2 +1 = 0\]

    \begin{Verbatim}[commandchars=\\\{\}]
{\color{incolor}In [{\color{incolor}123}]:} \PY{k}{function} \PY{n}{f!}\PY{p}{(}\PY{n}{F}\PY{p}{,} \PY{n}{x}\PY{p}{)}
              \PY{n}{F}\PY{p}{[}\PY{l+m+mi}{1}\PY{p}{]} \PY{o}{=} \PY{l+m+mi}{2}\PY{o}{*}\PY{n}{x}\PY{p}{[}\PY{l+m+mi}{1}\PY{p}{]} \PY{o}{+} \PY{n}{x}\PY{p}{[}\PY{l+m+mi}{3}\PY{p}{]}
              \PY{n}{F}\PY{p}{[}\PY{l+m+mi}{2}\PY{p}{]} \PY{o}{=} \PY{l+m+mf}{1.0} \PY{o}{\PYZhy{}} \PY{n}{x}\PY{p}{[}\PY{l+m+mi}{3}\PY{p}{]} \PY{o}{*} \PY{o}{\PYZhy{}}\PY{l+m+mi}{2} \PY{o}{*} \PY{n}{x}\PY{p}{[}\PY{l+m+mi}{2}\PY{p}{]}
              \PY{n}{F}\PY{p}{[}\PY{l+m+mi}{3}\PY{p}{]} \PY{o}{=} \PY{o}{\PYZhy{}}\PY{n}{x}\PY{p}{[}\PY{l+m+mi}{1}\PY{p}{]} \PY{o}{\PYZhy{}} \PY{n}{x}\PY{p}{[}\PY{l+m+mi}{2}\PY{p}{]}\PY{o}{\PYZca{}}\PY{l+m+mf}{2.0} \PY{o}{+} \PY{l+m+mf}{1.0}
          \PY{k}{end}
          
          \PY{n}{x0} \PY{o}{=} \PY{p}{[}\PY{o}{\PYZhy{}}\PY{l+m+mf}{2.37}\PY{p}{;} \PY{o}{\PYZhy{}}\PY{l+m+mf}{1.836}\PY{p}{;} \PY{o}{\PYZhy{}}\PY{l+m+mf}{1.0}\PY{p}{]}
          \PY{n}{sol} \PY{o}{=} \PY{n}{nlsolve}\PY{p}{(}\PY{n}{f!}\PY{p}{,} \PY{n}{x0}\PY{p}{)}
          \PY{n}{x1} \PY{o}{=} \PY{n}{sol}\PY{o}{.}\PY{n}{zero}\PY{p}{[}\PY{l+m+mi}{1}\PY{p}{]}
          \PY{n}{x2} \PY{o}{=} \PY{n}{sol}\PY{o}{.}\PY{n}{zero}\PY{p}{[}\PY{l+m+mi}{2}\PY{p}{]}
          \PY{n}{λ₁} \PY{o}{=} \PY{n}{sol}\PY{o}{.}\PY{n}{zero}\PY{p}{[}\PY{l+m+mi}{3}\PY{p}{]}
          \PY{n}{println}\PY{p}{(}\PY{l+s}{\PYZdq{}}\PY{l+s}{x}\PY{l+s}{₁}\PY{l+s}{ }\PY{l+s}{=}\PY{l+s}{ }\PY{l+s+si}{\PYZdl{}x1}\PY{l+s}{\PYZdq{}}\PY{p}{)}
          \PY{n}{println}\PY{p}{(}\PY{l+s}{\PYZdq{}}\PY{l+s}{x}\PY{l+s}{₂}\PY{l+s}{ }\PY{l+s}{=}\PY{l+s}{ }\PY{l+s+si}{\PYZdl{}x2}\PY{l+s}{\PYZdq{}}\PY{p}{)}
          \PY{n}{println}\PY{p}{(}\PY{l+s}{\PYZdq{}}\PY{l+s}{λ}\PY{l+s}{₁}\PY{l+s}{ }\PY{l+s}{=}\PY{l+s}{ }\PY{l+s+si}{\PYZdl{}λ₁}\PY{l+s}{\PYZdq{}}\PY{p}{)}
          \PY{n}{println}\PY{p}{(}\PY{l+s}{\PYZdq{}}\PY{l+s}{T}\PY{l+s}{h}\PY{l+s}{e}\PY{l+s}{ }\PY{l+s}{p}\PY{l+s}{o}\PY{l+s}{t}\PY{l+s}{e}\PY{l+s}{n}\PY{l+s}{t}\PY{l+s}{i}\PY{l+s}{a}\PY{l+s}{l}\PY{l+s}{ }\PY{l+s}{o}\PY{l+s}{p}\PY{l+s}{t}\PY{l+s}{i}\PY{l+s}{m}\PY{l+s}{u}\PY{l+s}{m}\PY{l+s}{ }\PY{l+s}{p}\PY{l+s}{o}\PY{l+s}{i}\PY{l+s}{n}\PY{l+s}{t}\PY{l+s}{ }\PY{l+s}{i}\PY{l+s}{s}\PY{l+s}{ }\PY{l+s}{[}\PY{l+s+si}{\PYZdl{}x1}\PY{l+s}{,}\PY{l+s}{ }\PY{l+s+si}{\PYZdl{}x2}\PY{l+s}{]}\PY{l+s}{\PYZdq{}}\PY{p}{)}
          
          \PY{n}{lagr} \PY{o}{=} \PY{n}{hessf} \PY{o}{\PYZhy{}} \PY{n}{λ₁} \PY{o}{*} \PY{n}{hessg2}
          \PY{n}{isOpt} \PY{o}{=} \PY{n}{isposdef}\PY{p}{(}\PY{n}{lagr}\PY{p}{)}
          \PY{k}{if} \PY{n}{isOpt}
              \PY{n}{println}\PY{p}{(}\PY{l+s}{\PYZdq{}}\PY{l+s}{T}\PY{l+s}{h}\PY{l+s}{e}\PY{l+s}{ }\PY{l+s}{l}\PY{l+s}{a}\PY{l+s}{g}\PY{l+s}{r}\PY{l+s}{a}\PY{l+s}{n}\PY{l+s}{g}\PY{l+s}{i}\PY{l+s}{a}\PY{l+s}{n}\PY{l+s}{ }\PY{l+s}{i}\PY{l+s}{s}\PY{l+s}{ }\PY{l+s}{p}\PY{l+s}{o}\PY{l+s}{s}\PY{l+s}{i}\PY{l+s}{t}\PY{l+s}{i}\PY{l+s}{v}\PY{l+s}{e}\PY{l+s}{ }\PY{l+s}{d}\PY{l+s}{e}\PY{l+s}{f}\PY{l+s}{i}\PY{l+s}{n}\PY{l+s}{i}\PY{l+s}{t}\PY{l+s}{e}\PY{l+s}{ }\PY{l+s}{t}\PY{l+s}{h}\PY{l+s}{e}\PY{l+s}{r}\PY{l+s}{e}\PY{l+s}{f}\PY{l+s}{o}\PY{l+s}{r}\PY{l+s}{e}\PY{l+s}{ }\PY{l+s}{t}\PY{l+s}{h}\PY{l+s}{e}\PY{l+s}{ }\PY{l+s}{p}\PY{l+s}{o}\PY{l+s}{t}\PY{l+s}{e}\PY{l+s}{n}\PY{l+s}{t}\PY{l+s}{i}\PY{l+s}{a}\PY{l+s}{l}\PY{l+s}{ }\PY{l+s}{o}\PY{l+s}{p}\PY{l+s}{t}\PY{l+s}{i}\PY{l+s}{m}\PY{l+s}{u}\PY{l+s}{m}\PY{l+s}{ }\PY{l+s}{i}\PY{l+s}{s}\PY{l+s}{ }\PY{l+s}{a}\PY{l+s}{ }\PY{l+s}{c}\PY{l+s}{o}\PY{l+s}{n}\PY{l+s}{s}\PY{l+s}{t}\PY{l+s}{r}\PY{l+s}{a}\PY{l+s}{i}\PY{l+s}{n}\PY{l+s}{e}\PY{l+s}{d}\PY{l+s}{ }\PY{l+s}{o}\PY{l+s}{p}\PY{l+s}{t}\PY{l+s}{i}\PY{l+s}{m}\PY{l+s}{u}\PY{l+s}{m}\PY{l+s}{\PYZdq{}}\PY{p}{)}
          \PY{k}{end}
\end{Verbatim}


    \begin{Verbatim}[commandchars=\\\{\}]
x₁ = -0.2258029814778883
x₂ = -1.1071598716887676
λ₁ = 0.4516059629557766
The potential optimum point is [-0.2258029814778883, -1.1071598716887676]
The lagrangian is positive definite therefore the potential optimum is a constrained optimum

    \end{Verbatim}

    The point {[}-0.2258, -1.1072 is the constrained optimum for the problem
without the equality constraint.


    % Add a bibliography block to the postdoc
    
    
    
    \end{document}
